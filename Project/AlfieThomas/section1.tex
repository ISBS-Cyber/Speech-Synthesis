\begin{frame}{MND and Speech Impairment}
    \begin{block}{About MND \cite{NHS2025}}
        \begin{itemize}
            \item MND or Motor Neuron Disease is an incurable disease. %that can affect anyone at any age, although it typically affects those older than 50. 
            \item It is progressive and eventually fatal. %however progression varies from person to person. 
            \item It occurs when the motor neurons in a person's body stop working correctly.
            \item It affects:
            \begin{itemize}
                \item Mobility and movement.
                \item Breathing, swallowing, eating and drinking.
                \item Emotions, feelings, thoughts and behavior. 
                \item Speech and communication (focus for today).
            \end{itemize}
        \end{itemize}
    \end{block}
    \begin{block}{How it affects speech}
        MND can cause speech to become faint, slurred, or unclear. \cite{Leveque2022} Speech synthesis systems provide an alternative to those losing their speech. %eventually speech becomes impossible. 
    \end{block}
\end{frame}

\begin{frame}{How does Speech Synthesis Work?}
% In the past, people who were loosing their voice would use methods such as Message Banking, Voice Banking, and Digital Legacy to preserve their voice
There were two main problems with voice replacement in the past:
\begin{itemize}
    \item Limited in Scope
    \item Unusable for people who never had a voice

%\begin{figure}
%   \centering
%   \includegraphics[width=0.5\linewidth]{NeuralTTS.png}
%   % \caption{Enter Caption}
%   \label{fig:placeholder}
%\end{figure}
  \item Modern Speech Synthesis is made up of three key components \cite{tan2021survey}
    \begin{itemize}
        \item The Text Analysis module %that extracts linguistic features from the given text
        \item The Acoustic Model %that then translates them into acoustic features
        \item The Vocoder %that finally synthesizes the waveform from those features
    \end{itemize}
\end{itemize}
\end{frame}

\begin{frame}{How does this impact accessibility in the internet age?}
There are many incredible technologies being developed at the moment for accessibility, such as eye tracking \cite{elsahar2019augmentative}. %which allows the control of the system using just a user's eyes.

%\begin{figure}
%    \centering
%    \includegraphics[width=0.5\linewidth]{pic/eye_trakcing.png}
%    \label{fig:placeholder}
%\end{figure}

In the rare case that someone can't use either eye, such as from Sixth Nerve Palsy\cite{sixthNervePalsy}, there are still options like breath control.


This ensures speech synthesis is always a viable option.
\end{frame}