
\begin{frame}{How feasible is it to make an accurate voice representation with few voice samples?}

32 phonetically rich samples can mean a naturalness MOS of up to 3.54, and a speaker similarity MOS of 4.28 can be achieved.

No. of samples may be more than some affected individuals may actually have.

Limited means of reducing potential long silences and scratching sounds.

Can be somewhat improved by training with Human Feedback.

\end{frame}

\begin{frame}{"Is similarity to the affected individual's voice samples always ideal?"}

Given the nature of MND, an individual may be unlikely to have sufficient speech recordings.

Late sampling can result in “partial dysarthric pronunciation patterns incorporated into the reconstructed speech”.

Can be mitigated by fine-tuning, but there's no guarantee users will find product expressive.

There is a somewhat of a trade-off between naturalness and speaker similarity.

\end{frame}
