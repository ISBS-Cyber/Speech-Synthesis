\begin{frame}{How feasible is it to make an accurate voice representation with few voice samples?}
    \begin{block}{Training:}
        \begin{itemize}
            \item Five minutes of speech can yield a MOS as high as 3.97, and a speaker similarity MOS of 4.25 \cite{zhou2024thu}.
            \item No. of samples may be more than some affected individuals may actually have.
            \item Limited means of reducing potential long silences and scratching sounds \cite{peng2024voicecraft}.
            \item Can be somewhat improved by training with Human Feedback \cite{chen2024humanfeedback}.
        \end{itemize}
    \end{block}
\end{frame}

\begin{frame}{Challenge: Is similarity to the affected individual's voice samples always ideal?}
    \begin{block}{Difficulties:}
        \begin{itemize}
            \item Given the nature of MND, an individual may be unlikely to have sufficient speech recordings \cite{wang2022speaker}.
            \item Late sampling can result in “partial dysarthric pronunciation patterns incorporated into the reconstructed speech”.
            \item Can be mitigated by fine-tuning, but there's no guarantee users will find product expressive.
            \item There is somewhat of a trade-off between naturalness and speaker similarity.
        \end{itemize}
    \end{block}
\end{frame}

\begin{frame}{Challenge: Is speech synthesis reliably accessible to users?}
    \begin{block}{Ussability:}
        \begin{itemize}
            \item To use a speech synthesis model offline, a GPU is often needed \cite{mandeel2022speaker}.
            \item StreamSpeech uses just one CPU core, at cost of lower MCD \cite{shopov2023streamspeech}.
            \begin{itemize}
                \item Not appropriate for current use, but a proof of future concept.
            \end{itemize}
        \end{itemize}
    \end{block}
\end{frame}

\begin{frame}{Solution: Use of Concepts like Transfer Learning}
    \begin{block}{Transfer learning:}
        \begin{itemize}
            \item Transfer Learning is a process where knowledge gained through one task or dataset is used to enhance the models performance for a related task \cite{torrey2010transfer}. 
            \item  Hidden Markov Model (HMM) was standard for most early work in speech synthesis and voice banking.
            \item Neural TTS is newer, though it requires more data than HMM, it also possesses an increased naturalness.
        \end{itemize}
    \end{block}
\end{frame}

\begin{frame}{Solution: Development of Naturalness in Newer Models}
    \begin{block}{Naturalness development:}
        \begin{itemize}
            \item \textbf{Naturalness} - how closely a TTS speech mimics a humans, more qualitative measure than accuracy which is measured in MOS.
            \item \textbf{Tacotron 2} - has already been applied and used by those with MND and Dysarthria. 
            \item \textbf{Neural HMM} - proposed altered model of Tacotron 2, requires less data and has great potential for those with MND.
        \end{itemize}
    \end{block}
\end{frame}